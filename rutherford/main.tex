
\documentclass{article}
\usepackage[utf8]{inputenc}
\usepackage{blindtext}
\usepackage{graphicx}
\usepackage{amsmath}
\usepackage{csvsimple}
\usepackage{pdfpages}
\usepackage{hyperref}
\usepackage{gensymb}

\begin{document}
\begin{center}
\textbf{\Huge{University of South Bohemia}}\\
\vspace{50px}
\textbf{\Large{Faculty of Science}} \\
\vspace{30px}
\includegraphics[width=120px]{~/school/logo.png} \\
\vspace{30px}
\textbf{\large{Praktika IV}}
\vspace{20px}
\\
\vspace{20px}
\large{Frank-Hertzův experiment} \\
\vspace{60px}
\end{center}
\begin{flushleft}
Datum: 18.10.2023 \\
Jmeno: Martin Skok \\
Obor: Fyzika \\
Hodnoceni:
\end{flushleft}
\newpage
\section{Úkoly}
Změřte závislost počtu rozptýlených $\alpha$ částic na úhlu rozptylu
\section{Pomůcky}
Experimentální komora, zdroj $\alpha$ částic $^{241}A_{m}$, pumpa na vysátí vzduchu, hadičky, čítač atd...
\section{Teorie}
\section{Postup měření}
Můj vedoucí praktik zapnul měřák dávkového příkonu Radigem.
Do komory jsem vložil vzorek.
Měřící komoru jsem připojil k pumpě, aby jsem z ní mohl vysát vzduch a vytvořit vakuum.
Ke komoře jsem připojil k čítači, který ukazoval počet detekovaných částic.
Zapnul jsem čítač a nastavil ho do polohy $N_{A_E}$.
Rameno se vzorkem jsem otočil do polohy 30°, aby na detektor dopadalo minimální množství částic.
Zapnul jsem čítač a postupně zvyšoval diskriminační napětí, kdy šum klesl na nulovou hodnotu.
Zaznamenal jsem si hodnotu.
To samé jsem udělal pro úhel 0° a zaznamenal jsem si hodnotu.
Diskriminační napětí jsem potom nastavil na střed těchto hodnot.
Pomocí tlačítka GATE jsem nastavil čas na 100 sekund a měřil jsem četnost částic pro úhly 0° 5° -5° 10° -10° 15° -15°.
Podobně jsem měřil pro úhly 20° -20° (200s), 25° -25° (600s) a 30° -30° (900s).
\section{}
\section{}
\section{}
\end{document}
