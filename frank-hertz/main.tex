\documentclass{article}
\usepackage[utf8]{inputenc}
\usepackage{blindtext}
\usepackage{graphicx}
\usepackage{amsmath}
\usepackage{csvsimple}
\usepackage{pdfpages}
\usepackage{hyperref}

\begin{document}
\begin{center}
\textbf{\Huge{University of South Bohemia}}\\
\vspace{50px}
\textbf{\Large{Faculty of Science}} \\
\vspace{30px}
\includegraphics[width=120px]{~/school/logo.png} \\
\vspace{30px}
\textbf{\large{Praktika III}}
\vspace{20px}
\\
\vspace{20px}
\large{Frank-Hertzův experiment} \\
\vspace{60px}
\end{center}
\begin{flushleft}
Datum: 18.8.2023 \\
Jmeno: Martin Skok \\
Obor: Fyzika \\
Hodnoceni:
\end{flushleft}
\newpage
\section{Úkoly}
\subsection{}
Proměřit voltampérovou chrakteristiku Franck-Hertzovy trubice (triody) plněné
rtuťovými parami a určit energii přechodu v atomu rtuti.
\subsection{}
Proměřit voltampérovou chrakteristiku Franck-Hertzovy trubice (tetrody) plněné
neonem a určit energii přechodu v atomu neonu.
\subsection{}
Přepočíst zjištěné hodnoty energií $\Delta E [eV]$ na vlnočet $\nu [cm^{-1}]$ a po porovnání
se spektroskopickými tabulkami určit atomové orbitaly účastnící se naměřených energetických
přechodů v atomech rtuťi a neonu.
\section{Pomůcky}
Franck-Hertzova trubice plněná rtuťovými parami, Franck-Hertzova trubice plněná neonem,
napájecí zdroj, zařízení 3BNETlog, počítač s aplikací 3BNETlog a propojovací vodiče.
\section{Teorie}
\end{document}
