% https://cs.wikipedia.org/wiki/Rydbergova_konstanta
% https://cs.wikipedia.org/wiki/Balmerova_s%C3%A9rie
% https://chem.libretexts.org/Bookshelves/General_Chemistry/Book%3A_General_Chemistry%3A_Principles_Patterns_and_Applications_(Averill)/06%3A_The_Structure_of_Atoms/6.03%3A_Atomic_Spectra_and_Models_of_the_Atom

\documentclass{article}
\usepackage[utf8]{inputenc}
\usepackage{blindtext}
\usepackage{graphicx}
\usepackage{amsmath}
\usepackage{csvsimple}
\usepackage{pdfpages}
\usepackage{hyperref}
\usepackage{gensymb}

\begin{document}
\begin{center}
\textbf{\Huge{University of South Bohemia}}\\
\vspace{50px}
\textbf{\Large{Faculty of Science}} \\
\vspace{30px}
\includegraphics[width=120px]{~/school/logo.png} \\
\vspace{30px}
\textbf{\large{Praktika IV}}
\vspace{20px}
\\
\vspace{20px}
\large{Určení Rydbergovi konstanty} \\
\vspace{60px}
\end{center}
\begin{flushleft}
Datum: 18.10.2023 \\
Jmeno: Martin Skok \\
Obor: Fyzika \\
Hodnoceni:
\end{flushleft}
\newpage
\section{Úkoly}
\begin{itemize}
        \item Okalibrujte spektograf pomocí sodíkové lampy
        \item Změřte vlnové délky čar Balmerovy série vodíku/deuteria
        \item Vypočtěte Rydbergovu konstantu a z ní vypočtěte Planckovu konstantu
\end{itemize}

\section{Seznam pomůcek}
Spektrometr, držák s optickou mřížkou/hranolem, sodíková lampa, zdroj sodíkové lampy,
vodíková spektrální trubice, lampička na přisvícení.
\section{Teorie}
Rydbergova konstanta je fyzikální konstanta pojmenovaná po švédském fyzikovi Johannesu Rydbergovi. Představuje nejvyšší možný vlnočet (převrácená hodnota vlnové délky) elektromagnetického záření, které může vyzářit nejjednodušší atom – atom vodíku – v limitě nekonečné hmotnosti jádra.\\
V tomto měření chceme najít Rydbergovu konstantu z Balmerovi série.\\
Balmerova série je série spektrálních čar (Balmerovy čáry) ve spektru atomů vodíku, které vznikají při přechodu elektronů mezi druhou energetickou hladinou a vyššími hladinami. Při přechodu elektronu mezi druhou a třetí energetickou hladinou se utváří při astronomických pozorování velmi důležitá červená čára $H_{\alpha}$ (s vlnovou délkou 656,3 nm), při přechodu mezi druhou a čtvrtou hladinou vzniká čára $H_{\beta}$ atd...
\begin{equation}
  \frac{1}{\lambda_{n}} = R_{H} \left( \frac{1}{2^{2}} - \frac{1}{n^{2}} \right) \left[\frac{1}{m}\right] ; \quad (n \in N) \land (n > 2)
\end{equation}
$\lambda$ je vlnová délka a $R_{H}$ je Rydbergova konstanta pro vodík.\\
Klasickou korekci Rydbergovi konstanty na konečné jádro provedeme pomocí vzorce
\begin{equation}
  R_{\infty} = R_{H} \frac{m_{p} + m_{e}}{m_{p}}
\end{equation}
$m_{p} = 1.6726 \cdot 10^{-27}[kg]$ je hmotnost protonu a $m_{e} = 9.1093 \cdot 10^{-31}[kg]$ je hmotnost elektronu.\\
Polohy maxim jsou dány vzorcem
\begin{equation}
  d sin \theta = m \lambda
\end{equation}
$d$ je mřížková konstanta, $m$ je řád difrakce, $\lambda$ je vlnová délka a $\theta$ je úhel difrakce.
Úhel pro difrakci se zjistí ze vztahu
\begin{equation}
  \theta = \frac{\theta_{L} - \theta_{P}}{2}
\end{equation}
$\theta_{L}$ je úhel naměřený vlevo od nulové polohy a $\theta_{R}$ úhel naměřený vpravo od nulové polohy.
\section{Postup měření}
\subsection{Kalibrace spektrometru a měření maxim u sodíku}
Když jsem přišel, vše bylo zapojeno, zkontroloval jsem tedy zapojení.
Zapl jsem sodíkovou lampu a přesunu tak, aby svítila na štěrbinu.
Nastavil jsem spektrometr tak, aby byl v jedné přímce s trubicí.
Potom jsem nastavil velikost štěrbiny a zaostřil jsem dalekohled.
Nastavil jsem dalekohled, aby první čára ze spektra byla přesně na kříži.
Zaznamenal jsem si tuto hodnotu.
Posouval jsem dalekohled doleva, dokud jsem neobjevil dalši čáru
a zapsal jsem si její hodnotu vycentrovanou na kříži.
To jsem udělal ještě pro jednu a vrátil jsem dalekohled do nulové polohy.
Opakoval jsem následující i pro pravou stranu.
\subsection{Pozorování Balmerovy série}
Vyměnil jsem sodíkovou lampu za vodíkovou.
Nastavil jsem rameno do nulové polohy a dělal jsem přesně to samý jako u sodíkový lampy.
Dokázal jsem změřit difrakci k 4. řádu.
\section{}
\section{}
\section{}
\section{}
\section{}
\end{document}
