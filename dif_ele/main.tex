% https://virtuelle-experimente.de/en/b-feld/b-feld/versuchsaufbau.php
\documentclass{article}
\usepackage[utf8]{inputenc}
\usepackage{blindtext}
\usepackage{graphicx}
\usepackage{amsmath}
\usepackage{csvsimple}
\usepackage{pdfpages}
\usepackage{hyperref}
\usepackage{gensymb}

\begin{document}
\begin{center}
\textbf{\Huge{University of South Bohemia}}\\
\vspace{50px}
\textbf{\Large{Faculty of Science}} \\
\vspace{30px}
\includegraphics[width=120px]{~/school/logo.png} \\
\vspace{30px}
\textbf{\large{Praktika IV}}
\vspace{20px}
\\
\vspace{20px}
\large{Comptnův rozptyl} \\
\vspace{60px}
\end{center}
\begin{flushleft}
Datum: 18.10.2023 \\
Jmeno: Martin Skok \\
Obor: Fyzika \\
Hodnoceni:
\end{flushleft}
\newpage
\section{Úkoly}
\begin{itemize}
  \item Změřte specifický náboj elektronu
  \item Ze známé hodnoty elementárního náboje vypočtěte hmotnost elektronu
\end{itemize}
\section{Seznam pomůcek}
Thompsonova Trubice,
Helmholtzovy cívky,
vysokonapěťový zdroj,
proudový zdroj,
posuvné měřítko,
luminiscenční deska,
propojovací vodiče,
elektronová difrakční trubice
\section{Teorie}
Helmholtzovy cívky tvoři homogení magnetické pole.
Uvnitř těchto cívek je katodová trubice která generuje paprsek elektronů.
Směr pohybu elektronu je kolmý k magnetickému poli.
Aby byly trajektrorie elektronů vidět, experiment se provede
ve skleněné nádobě naplněné neonovým plynem.



\section{Postup měření}
Nejdříve jsem zapojil měření s Helmholtovými cívkami a všechny potenciometry jsem nastavil na nulu.
Zapnul jsem vysokonapěťový zdroj a počkal jsem, až se katoda nažhavý.
Nastavil jsem urychlovací na 2 kV. Na luminiscenční desce byl vidět svazek.
Zapnul jsem porudový zdroj a zvyšoval proud, dokud svazek nedosáhl stupnice na hodnotě $40 mm$.
Zvyšoval jsem proud v cívkách a zaznamenával hodnoty pro různé hodnoty ohybu.
Potom jsem zvíšil napětí na 3kV a opakoval měření proudu.
To jsem zopáknul pro hodnoty 4kV, 5kV.
\section{}
\section{}
\section{}
\section{}
\end{document}
