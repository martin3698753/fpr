% https://virtuelle-experimente.de/en/b-feld/b-feld/versuchsaufbau.php
\documentclass{article}
\usepackage[utf8]{inputenc}
\usepackage{blindtext}
\usepackage{graphicx}
\usepackage{amsmath}
\usepackage{csvsimple}
\usepackage{pdfpages}
\usepackage{hyperref}
\usepackage{gensymb}

\begin{document}
\begin{center}
\textbf{\Huge{University of South Bohemia}}\\
\vspace{50px}
\textbf{\Large{Faculty of Science}} \\
\vspace{30px}
\includegraphics[width=120px]{~/school/logo.png} \\
\vspace{30px}
\textbf{\large{Praktika IV}}
\vspace{20px}
\\
\vspace{20px}
\large{Difrakce elektronů} \\
\vspace{60px}
\end{center}
\begin{flushleft}
Datum: 22.12.2023 \\
Jmeno: Martin Skok \\
Obor: Fyzika \\
Hodnoceni:
\end{flushleft}
\newpage
\section{Úkoly}
\begin{itemize}
  \item Změřte specifický náboj elektronu
  \item Ze známé hodnoty elementárního náboje vypočtěte hmotnost elektronu
  \item Pozorujte difrakci elektronového svazku na stínítku baňky
  \item Určete vzdálenosti mezi rovinami grafitové difrakční mřížky
\end{itemize}
\section{Seznam pomůcek}
Thompsonova Trubice,
Helmholtzovy cívky,
vysokonapěťový zdroj,
proudový zdroj,
posuvné měřítko,
luminiscenční deska,
propojovací vodiče,
elektronová difrakční trubice
\section{Teorie}
\subsection{Specifický náboj elektronu}
Helmholtzovy cívky tvoři homogení magnetické pole.
Uvnitř těchto cívek je katodová trubice která generuje paprsek elektronů.
Směr pohybu elektronu je kolmý k magnetickému poli.
Aby byly trajektrorie elektronů vidět, experiment se provede
ve skleněné nádobě naplněné neonovým plynem.
Intenzita magnetického pole je dána jako
\begin{equation}\label{eq:B}
  B = \left( \frac{4}{5} \right)^{\frac{3}{2}} \frac{\mu_{0}N}{R} I
\end{equation}
Kde $N$ je počet závitů, což bude v našem případě 320 a $R$ je poloměr cívky, což je v našem případě
$62 mm$.\\
Poloměr zakřivení dráhy elektronů bude pro tento případ. Vše se odvodí z geometri problému.
\begin{equation}\label{eq:r}
  r = \frac{(80mm)^{2} + e^{2}}{\sqrt{2} (80mm - e)}
\end{equation}
A konečně specifický náboj pak získáme pomocí
\begin{equation}\label{eq:qm}
  \frac{q}{m} = \frac{2V}{(Br)^{2}}
\end{equation}
Kde $V$ je napětí.
\subsection{Difrakce elektronu}
Podle L.D.Broglieho mají elektrony a další částice vlnovou délku, která je nepřímo úměrná jejich
hybnosti.
\begin{equation}\label{eq:deb}
  \lambda = \frac{h}{p}
\end{equation}
Díky tomu mají rychlé elektrony velmi krátkou vlnovou délku, která je srovnatelná s rozestupy mezi atomovými vrstvami v krystalech. Protože elektrony mají vlnovou povahu, mohou podléhat difrakci.
Kinetická energie elektronu je
\begin{equation}\label{eq:kin}
  T = eV = \frac{p^{2}}{2m_{e}}
\end{equation}
Vyjádříme hybnost a dosadíme do rovnice nahoře
\begin{equation}\label{eq:lam}
  \lambda = \frac{h}{\sqrt{2m_{e}eV}}
\end{equation}
Z difrakce na jedné štěrbině vím, že
\begin{equation}\label{eq:dif}
  2dsin(\alpha) = n \lambda
\end{equation}
Z geometri problému můžeme vyjádřit úhel z této rovnice
\begin{equation}\label{eq:tan}
  tan(2 \alpha) = \frac{\frac{D}{2}}{l_{1} + l_{2}}
\end{equation}
Kde $l_{1} = L - R$ a $l_{2} = \sqrt{R^{2} - \frac{D^{2}}{4}}$\\
$L = 135mm$ a $R = 65mm$ jsou paramtery trubice.\\
To vše pak můžeme dát dohromady a dostaneme tuto rovnici:
\begin{equation}\label{eq:final}
  d = n \lambda \cdot \frac{1}{2sin \left(\frac{1}{2} arctan \left( \frac{D}{2(l_{1} + l_{2})} \right)  \right)}
\end{equation}


\section{Postup měření}
\subsection{Specifický náboj elektronu}
Nejdříve jsem zapojil měření s Helmholtovými cívkami a všechny potenciometry jsem nastavil na nulu.
Zapnul jsem vysokonapěťový zdroj a počkal jsem, až se katoda nažhavý.
Nastavil jsem urychlovací na 2 kV. Na luminiscenční desce byl vidět svazek.
Zapnul jsem porudový zdroj a zvyšoval proud, dokud svazek nedosáhl stupnice na hodnotě $40 mm$.
Zvyšoval jsem proud v cívkách a zaznamenával hodnoty pro různé hodnoty ohybu.
Potom jsem zvíšil napětí na 3kV a opakoval měření proudu.
To jsem zopáknul pro hodnoty 4kV, 5kV.
\subsection{Difrakce elektronu}
Před zapnutím jsem stáhnul zdroj na nulu. Zvyšoval jsem napětí na $4KV$. Na skle jsem potom viděl
dva prstence. Pomocí posuvného měřítka jsem změřil průměry obou prstenců. Toto jsem udělal i pro další napětí do $5KV$.
\section{Data}
\subsection{Specifický náboj elektronu}
\csvreader[
tabular = |c|c|c|c|c|,
table head =
\hline
{} & {$2KV$} & {$3KV$} & {$4KV$} & {$5KV$}\\
{ohyb $e[mm]$} & {Proud $I[A]$} & {Proud $I[A]$} & {Proud $I[A]$} & {Proud $I[A]$}\\
\hline
\hline,
late after line = \\\hline
]{data/coils_cur.csv}{}{
  \csvcoli & \csvcolii & \csvcoliii & \csvcoliv & \csvcolv}
\\
\vspace{1em}
\\
Specifký náboj elektronu jsem počítal podle rovnice \ref{eq:qm}.
\\
\vspace{1em}
\\
\csvreader[
tabular = |c|c|c|c|,
table head =
\hline
{$2KV$} & {$3KV$} & {$4KV$} & {$5KV$}\\
\hline
\multicolumn{4}{|c|}{Specifický náboj $\frac{q}{m} \left[ \frac{C}{kg} \cdot 10^{11} \right]$}\\
\hline
\hline,
late after line = \\\hline
]{data/coils_charge.csv}{}{
  \csvcoli & \csvcolii & \csvcoliii & \csvcoliv}
\\
\vspace{1em}
\\
Aritmetický průměr:
$$\overline{ \left( \frac{q}{m} \right)} = 1.545288 \cdot 10^{11}$$
Směrodatná odchylka:
$$\sigma_{\frac{q}{m}} = 0.083 \cdot 10^{11}$$
\vspace{1em}
Teď můžeme vypočítat hmotnost elektronu ze známého náboje elektronu
$$m_{e} = \frac{e}{\frac{q}{m}} = \frac{1.602177 \cdot 10^{(-19)}}{1.545288 \cdot 10^{11}} = 10.3682 \cdot 10^{-31}$$
$$\sigma_{m_{e}} = 0.019326 \cdot 10^{-31}$$
\subsection{Difrakce elektronu}
\csvreader[
tabular = |c|c|c|,
table head =
\hline
{Napětí $V[KV]$} & {$D_1[mm]$} & {$D_2[mm]$}\\
\hline
\hline,
late after line = \\\hline
]{data/rings.csv}{}{
  \csvcoli & \csvcolii & \csvcoliii}
\\
\vspace{1em}
\\
Vzdálenosti mezi rovinami grafitové difrakční mřížky jsem získal ze vztahu \ref{eq:final}.
\\
\vspace{1em}
\\
\csvreader[
tabular = |c|c|c|,
table head =
\hline
{Napětí $V[KV]$} & {$d_1[pm]$} & {$d_2[pm]$}\\
\hline
\hline,
late after line = \\\hline
]{data/dif_circle.csv}{}{
  \csvcoli & \csvcolii & \csvcoliii}
\\
\vspace{1em}
\\
$$\overline{d_1} = 196.862 \cdot 10^{12}$$
$$\sigma_{d_1} = 6.295 \cdot 10^{12}$$
$$\overline{d_2} = 119.327 \cdot 10^{12}$$
$$\sigma_{d_2} = 1.843 \cdot 10^{12}$$
\section{Diskuse}
Úkolem bylo určit specifický náboj a následně hmostnost elektronu.
Potom jsem měl určit vzdálenosti rovin v krystalové mřížce.
Výsledné hodnoty se od tabulkových lišil i mimo rozsah chyb. \cite{r:qm} \cite{r:em} \\
Chyby mohly být způsobeny nepřesným změřeným vzdáleností pomocí posuvného měřítka.
Taky elektronové svazky které jsem pozoroval a pak odečítal na stupnici bylo velmi nepřesné.
Činil jsem tak pouze svým okem a mím ''smyslem o přesnosti''.
\section{Závěr}
Naměřené hodnoty:\\
Specifiký náboj
$$\frac{q}{m} = (1.545 \pm 0.083) \cdot 10^{11} \left[ \frac{C}{kg} \right]$$
Hmotnost elektronu
$$m_{e} = (10.368 \pm 0.019) \cdot 10^{-31} [kg]$$
vzdálenost rovin krystalové mřížky grafitu
$$d_{1} = (196.862 \pm 6.295) \cdot 10^{12} [m]$$
$$d_{2} = (119.327 \pm 1.843) \cdot 10^{12} [m]$$
\begin{thebibliography}{1}
  \bibitem{r:qm}
  \url{https://en.wikipedia.org/wiki/Mass-to-charge_ratio}
  \bibitem{r:em}
  \url{https://en.wikipedia.org/wiki/Electron_mass}
  \bibitem{r:hc}
  \url{https://virtuelle-experimente.de/en/b-feld/b-feld/versuchsaufbau.php}
\end{thebibliography}
\end{document}
