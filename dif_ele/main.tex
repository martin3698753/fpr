% https://virtuelle-experimente.de/en/b-feld/b-feld/versuchsaufbau.php
\documentclass{article}
\usepackage[utf8]{inputenc}
\usepackage{blindtext}
\usepackage{graphicx}
\usepackage{amsmath}
\usepackage{csvsimple}
\usepackage{pdfpages}
\usepackage{hyperref}
\usepackage{gensymb}

\begin{document}
\begin{center}
\textbf{\Huge{University of South Bohemia}}\\
\vspace{50px}
\textbf{\Large{Faculty of Science}} \\
\vspace{30px}
\includegraphics[width=120px]{~/school/logo.png} \\
\vspace{30px}
\textbf{\large{Praktika IV}}
\vspace{20px}
\\
\vspace{20px}
\large{Comptnův rozptyl} \\
\vspace{60px}
\end{center}
\begin{flushleft}
Datum: 18.10.2023 \\
Jmeno: Martin Skok \\
Obor: Fyzika \\
Hodnoceni:
\end{flushleft}
\newpage
\section{Úkoly}
\begin{itemize}
  \item Změřte specifický náboj elektronu
  \item Ze známé hodnoty elementárního náboje vypočtěte hmotnost elektronu
\end{itemize}
\section{Seznam pomůcek}
Thompsonova Trubice,
Helmholtzovy cívky,
vysokonapěťový zdroj,
proudový zdroj,
posuvné měřítko,
luminiscenční deska,
propojovací vodiče,
elektronová difrakční trubice
\section{Teorie}
\subsection{Specifický náboj elektronu}
Helmholtzovy cívky tvoři homogení magnetické pole.
Uvnitř těchto cívek je katodová trubice která generuje paprsek elektronů.
Směr pohybu elektronu je kolmý k magnetickému poli.
Aby byly trajektrorie elektronů vidět, experiment se provede
ve skleněné nádobě naplněné neonovým plynem.
Intenzita magnetického pole je dána jako
\begin{equation}
  B = \left( \frac{4}{5} \right)^{\frac{3}{2}} \frac{\mu_{0}N}{R} I
\end{equation}
Kde $N$ je počet závitů, což bude v našem případě 320 a $R$ je poloměr cívky, což je v našem případě
$62 mm$.\\
Poloměr zakřivení dráhy elektronů bude pro tento případ. Vše se odvodí z geometri problému.
\begin{equation}
  r = \frac{(80mm)^{2} + e^{2}}{\sqrt{2} (80mm - e)}
\end{equation}
A konečně specifický náboj pak získáme pomocí
\begin{equation}
  \frac{q}{m} = \frac{2V}{(Br)^{2}}
\end{equation}
Kde $V$ je napětí.
\subsection{Difrakce elektronu}
Podle L.D.Broglieho mají elektrony a další částice vlnovou délku, která je nepřímo úměrná jejich
hybnosti.
\begin{equation}
  \lambda = \frac{h}{p}
\end{equation}
Díky tomu mají rychlé elektrony velmi krátkou vlnovou délku, která je srovnatelná s rozestupy mezi atomovými vrstvami v krystalech. Protože elektrony mají vlnovou povahu, mohou podléhat difrakci.
Kinetická energie elektronu je
\begin{equation}
  T = eV = \frac{p^{2}}{2m_{e}}
\end{equation}
Vyjádříme hybnost a dosadíme do rovnice nahoře
\begin{equation}
  \lambda = \frac{h}{\sqrt{2m_{e}eV}}
\end{equation}
Z difrakce na jedné štěrbině vím, že
\begin{equation}
  2dsin(\alpha) = n \lambda
\end{equation}
Z geometri problému můžeme vyjádřit úhel z této rovnice
\begin{equation}
  tan(2 \alpha) = \frac{\frac{D}{2}}{l_{1} + l_{2}}
\end{equation}
Kde $l_{1} = L - R$ a $l_{2} = \sqrt{R^{2} - \frac{D^{2}}{4}}$\\
$L = 135mm$ a $R = 65mm$ jsou paramtery trubice.\\
To vše pak můžeme dát dohromady a dostaneme tuto rovnici:
\begin{equation}\label{final}
  d = n \lambda \cdot \frac{1}{2sin \left(\frac{1}{2} arctan \left( \frac{D}{2(l_{1} + l_{2})} \right)  \right)}
\end{equation}


\section{Postup měření}
\subsection{Specifický náboj elektronu}
Nejdříve jsem zapojil měření s Helmholtovými cívkami a všechny potenciometry jsem nastavil na nulu.
Zapnul jsem vysokonapěťový zdroj a počkal jsem, až se katoda nažhavý.
Nastavil jsem urychlovací na 2 kV. Na luminiscenční desce byl vidět svazek.
Zapnul jsem porudový zdroj a zvyšoval proud, dokud svazek nedosáhl stupnice na hodnotě $40 mm$.
Zvyšoval jsem proud v cívkách a zaznamenával hodnoty pro různé hodnoty ohybu.
Potom jsem zvíšil napětí na 3kV a opakoval měření proudu.
To jsem zopáknul pro hodnoty 4kV, 5kV.
\subsection{Difrakce elektronu}
Před zapnutím jsem stáhnul zdroj na nulu. Zvyšoval jsem napětí na $4KV$. Na skle jsem potom viděl
dva prstence. Pomocí posuvného měřítka jsem změřil průměry obou prstenců. Toto jsem udělal i pro další napětí do $5KV$.
\section{}
\section{}
\section{}
\section{}
\end{document}
